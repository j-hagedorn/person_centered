\PassOptionsToPackage{unicode=true}{hyperref} % options for packages loaded elsewhere
\PassOptionsToPackage{hyphens}{url}
%
\documentclass[
]{book}
\usepackage{lmodern}
\usepackage{amssymb,amsmath}
\usepackage{ifxetex,ifluatex}
\ifnum 0\ifxetex 1\fi\ifluatex 1\fi=0 % if pdftex
  \usepackage[T1]{fontenc}
  \usepackage[utf8]{inputenc}
  \usepackage{textcomp} % provides euro and other symbols
\else % if luatex or xelatex
  \usepackage{unicode-math}
  \defaultfontfeatures{Scale=MatchLowercase}
  \defaultfontfeatures[\rmfamily]{Ligatures=TeX,Scale=1}
\fi
% use upquote if available, for straight quotes in verbatim environments
\IfFileExists{upquote.sty}{\usepackage{upquote}}{}
\IfFileExists{microtype.sty}{% use microtype if available
  \usepackage[]{microtype}
  \UseMicrotypeSet[protrusion]{basicmath} % disable protrusion for tt fonts
}{}
\makeatletter
\@ifundefined{KOMAClassName}{% if non-KOMA class
  \IfFileExists{parskip.sty}{%
    \usepackage{parskip}
  }{% else
    \setlength{\parindent}{0pt}
    \setlength{\parskip}{6pt plus 2pt minus 1pt}}
}{% if KOMA class
  \KOMAoptions{parskip=half}}
\makeatother
\usepackage{xcolor}
\IfFileExists{xurl.sty}{\usepackage{xurl}}{} % add URL line breaks if available
\IfFileExists{bookmark.sty}{\usepackage{bookmark}}{\usepackage{hyperref}}
\hypersetup{
  pdftitle={A Person-Centered Framework},
  pdfauthor={Various authors},
  pdfborder={0 0 0},
  breaklinks=true}
\urlstyle{same}  % don't use monospace font for urls
\usepackage{longtable,booktabs}
% Allow footnotes in longtable head/foot
\IfFileExists{footnotehyper.sty}{\usepackage{footnotehyper}}{\usepackage{footnote}}
\makesavenoteenv{longtable}
\usepackage{graphicx,grffile}
\makeatletter
\def\maxwidth{\ifdim\Gin@nat@width>\linewidth\linewidth\else\Gin@nat@width\fi}
\def\maxheight{\ifdim\Gin@nat@height>\textheight\textheight\else\Gin@nat@height\fi}
\makeatother
% Scale images if necessary, so that they will not overflow the page
% margins by default, and it is still possible to overwrite the defaults
% using explicit options in \includegraphics[width, height, ...]{}
\setkeys{Gin}{width=\maxwidth,height=\maxheight,keepaspectratio}
\setlength{\emergencystretch}{3em}  % prevent overfull lines
\providecommand{\tightlist}{%
  \setlength{\itemsep}{0pt}\setlength{\parskip}{0pt}}
\setcounter{secnumdepth}{5}
% Redefines (sub)paragraphs to behave more like sections
\ifx\paragraph\undefined\else
  \let\oldparagraph\paragraph
  \renewcommand{\paragraph}[1]{\oldparagraph{#1}\mbox{}}
\fi
\ifx\subparagraph\undefined\else
  \let\oldsubparagraph\subparagraph
  \renewcommand{\subparagraph}[1]{\oldsubparagraph{#1}\mbox{}}
\fi

% set default figure placement to htbp
\makeatletter
\def\fps@figure{htbp}
\makeatother

\usepackage{booktabs}
\usepackage[]{natbib}
\bibliographystyle{apalike}

\title{A Person-Centered Framework}
\author{Various authors}
\date{2020-03-25}

\begin{document}
\maketitle

{
\setcounter{tocdepth}{1}
\tableofcontents
}
\hypertarget{reason}{%
\chapter{Reason}\label{reason}}

Each person's ability to choose a better future and map their journey toward it: this is what person-centered planning enables. In order for services to effectively support a person in this process, they must be provided within the context of a person's goals. Orienting a broad and complex system to keep the person at the center requires a consistent, overarching framework. MDHHS BHDDA is working to support this person-centered orientation with the following strategy:

\textbf{Goal:} To develop \protect\hyperlink{bok}{a common body of knowledge} for person-centered planning,
mapped to \protect\hyperlink{policy}{relevant policies} and \protect\hyperlink{research}{research},
which will inform a \protect\hyperlink{curriculum}{shared curriculum}
and \protect\hyperlink{measure}{measurement framework}
to support \protect\hyperlink{pcpdca}{improved quality of life} for each person.

\hypertarget{pcpdca}{%
\chapter{Person-Centered Planning (Doing, Checking, Acting)}\label{pcpdca}}

In this document and in the work under development with MDHHS-BHDDA, the phrase \emph{person-centered planning} is used broadly, to encompass not only the initial planning process but also its implementation, monitoring, and refinement. The need for this definition to extend beyond the PCP meetings and to direct all services and supports is already recognized within state policy, which indicates that:\footnote{MDHHS BHDDA Person-Centered Planning Policy (June 5, 2017), p.~1}

\begin{quote}
\emph{through PCP, a person is engaged in decision-making, problem solving, monitoring progress, and making needed adjustments to goals and supports and services provided in a timely manner.}
\end{quote}

Since the intent of supports and services is to improve personal quality of life, practitioners can view the PCP process as similar to the \emph{Plan-Do-Check-Act} (PDCA) cycle, which involves each element of the broad scope of PCP defined above. Versions of the PDCA cycle have already been successfully incorporated into the supports and treatment planning process for people with varying conditions and needs, from intellectual and developmental disabilities, to mental illness, to physical health concerns.

\hypertarget{bok}{%
\chapter{A Common Body of Knowledge}\label{bok}}

A foundational effort has been working to develop a shared repository of key terms and concepts, their definitions, and how they are related to one another. This common language has the following key features:

\begin{itemize}
\tightlist
\item
  Relates all terminology back to the person, who is at the center
\item
  Includes concepts related to the PCP process, as broadly defined above, as well as common attributes for understanding the person\footnote{Note that the body of knowledge is not intended to classify services and supports.}
\item
  Promotes consistency in implementation and training, and the scalability of future development
\item
  Allows for change; the language can be extended as new concepts are identified.\footnote{This approach also requires that ideas claiming to be new must differentiate themselves from existing terms and concepts.}
\item
  Reduces confusion across various policies with inconsistent terminology and scope
\end{itemize}

\hypertarget{policy}{%
\chapter{Comprehensive Mapping to Policy}\label{policy}}

As with any important idea, person-centered planning has been discussed and debated for decades, leaving a vast body of policy, regulations, guidance, and explanations to sift through. While many of the basic ideas of person-centered planning are simple and commonsense, practitioners are also required to adhere to existing policies. With this in mind, the body of knowledge is being developed:

\begin{itemize}
\tightlist
\item
  Based on a broad scope of relevant state and federal policies identified by MDHHS-BHDDA.
\item
  Using natural-language processing techniques to identify and refine core terminology from the text of the identified policies
\item
  In a manner that allows MDHHS-BHDDA to identify whether new federal policies would align with the current implementation of PCP
\end{itemize}

\hypertarget{research}{%
\chapter{Informed by (and Informing) Research}\label{research}}

Some activities related to person-centered planning are addressed in existing research.\footnote{For instance, \emph{goals and planning}, \emph{feedback and monitoring}, and similar activities defined by the \href{https://www.humanbehaviourchange.org/resources/behavioural-science/25/description}{Behaviour Change Intervention Ontology (BCIO)} have related research available.} In these instances, our goal would be to connect-the-dots between research and practice by making this knowledge available. In order to remain aligned with national efforts\footnote{NQF's \href{http://www.qualityforum.org/WorkArea/linkit.aspx?LinkIdentifier=id\&ItemID=91382}{Person-Centered Planning and Practice Project}, references `\emph{a research agenda to advance and promote person-centered planning in LTSS}'} in this area, initial steps would include:

\begin{itemize}
\tightlist
\item
  mapping of person-centered planning concepts to researched interventions, where these exist
\item
  use of key concepts from the body of knowledge for literature review and meta-analyses of PCP-related practices, to build a base of best practices and evidence for effectiveness
\item
  identification of gaps in existing research knowledge related to PCP
\end{itemize}

\hypertarget{curriculum}{%
\chapter{Shared Training Curriculum}\label{curriculum}}

To translate this common body of knowledge into action, various audiences need to be trained in the core elements of person-centered practice, using the foundational concepts identified and defined above. Work in developing these trainings includes:

\begin{itemize}
\tightlist
\item
  Identification of key audiences
\item
  Evaluation of potential training modalities based on key features
\item
  Developing a standard base curriculum as well as specialty topics for specific audiences
\end{itemize}

\hypertarget{measure}{%
\chapter{Person-Centered Measurement Framework}\label{measure}}

If the entire system of services and supports is intended to be person-centered, its performance should be measured within a framework that is centered on the person. Since many existing quality measures and data collection systems were not developed with this in mind, it will be important to develop a larger framework within which existing measures can be situated. This allows the system to retain the quality measurement work that has been completed, while acknowledging gaps within that framework which need to be filled.

Ongoing work in this area would include:

\begin{itemize}
\tightlist
\item
  Developing a person-centered measurement framework
\item
  Conducting an inventory of available data assets at a state wide level
\item
  Classifying existing measurement and data collection efforts (e.g.~HEDIS, BH-TEDS, etc.) within the context of this framework
\end{itemize}

  \bibliography{book.bib,packages.bib}

\end{document}
